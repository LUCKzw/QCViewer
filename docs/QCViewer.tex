\documentclass{article}
\usepackage{algorithmic}
\usepackage{algorithm}
\begin{document}
\title{QCViewer}
\author{Alex Parent, Jacob Parker}
\maketitle

\section{Introduction}
QCViewer is a tool to assist in the design, visualization and simulation of quantum circuits.

\section{Users Manual}

\section{Technical Details}
\subsection{Simulation}
We use a sparse state vector based simulation method.  Each state is superposition is stored (sparsely) as a 32-bit integer representing the state in the computational basis
and a complex number representing its amplitude. Gates are applied to each basis state individually creating and destroying states in the superposition as needed.

With this method the memory use associated with storing the state vector is dependent on the amount of superposition in the system.
%uncomment here and in Makefile if there are citations
%\bibliographystyle{ieeetr}
%\bibliography{QCViewer.bib}

\end{document}
